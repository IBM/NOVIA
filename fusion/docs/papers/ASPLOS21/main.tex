\documentclass[pageno]{jpaper}

%replace XXX with the submission number you are given from the ASPLOS submission site.
\newcommand{\asplossubmissionnumber}{XXX}

\usepackage[normalem]{ulem}

\begin{document}

\title{CRUCIBLE: Computationally-RedUctible Complex Inline accelerataBLE units}

%\date{}
\maketitle

\thispagestyle{empty}

\begin{abstract}

Computer and system architecture is currently experiencing a revolution. The dramatic
increase in workload variety and the usage of accelerators is shifting the paradigm from general-purpose
processors (GPP) providing the bulk of the performance, to more specialized units (i.e. accelerators) with tailored
design and specific to the task at which they excel. This has lead to the rise of new System-on-Chip (SoC) architectures
where GPP are complimented with a variety of heterogeneous accelerators. However, developing new hardware specific
for each application is a costly endeveour not only in development costs but also verification costs. More over, which
code region should be accelerated is of paramount importance to the designer, as each accelerator is tailored to a particular code region (i.e. functionality).
In this paper we present CRUCIBLE, an end-to-end methodology to derive Computationally-RedUctible Complex Inline accelerataBLE units
tailored to each application providing to the system designer an agile, systematic and cheap procedure to derive area-efficient by merging 
accelerator functionality, which does not only provide performance benefits with low area overhead, but is easily verifiable and highly integratable as they 
are part of the core.

\end{abstract}

\section{Introdution}

New and novel workloads require unprecedented performance from the systems in which 
they execute. Current trends focus on providing the needed performance through 
a tailored and specific computer architecture in form of accelerators that dramatically increases
the performance of workloads for which it has been design.

However, specialization of the architecture has several drawbacks. First, new
designs are costly since they require full teams of engineers identifying the computation
that needs acceleration, designing the system and finally validating the desgin. Moreover, 
the design of such accelerators is tailored to it's function, hence funelling funds and transistors to
a specific unit function rather than to a more general approach which could provide performance
to all types of workloads



\input{eval}
\section{Conclusions}
\label{sec:conc}


\section{Acknowledgements}

\bibliographystyle{plain}
\bibliography{references}


\end{document}

