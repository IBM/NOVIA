%%%%%%%%%%%%%%%%%%%%%%%%%%%%%%%%%%%%
% This is the template for submission to HPCA 2020
% The cls file is a modified from  'sig-alternate.cls'
%%%%%%%%%%%%%%%%%%%%%%%%%%%%%%%%%%%%

% https://www.hpca-conf.org/2020/wp-content/uploads/2019/07/hpca26_sample.pdf
% https://www.hpca-conf.org/2020/wp-content/uploads/2019/07/hpca26-latex-template.tar.gz

\documentclass{sig-alternate}
\setlength{\paperheight}{11in}
\setlength{\paperwidth}{8.5in}

\newcommand{\ignore}[1]{}
\usepackage[pass]{geometry}
\usepackage{fancyhdr}
\usepackage[normalem]{ulem}
\usepackage[hyphens]{url}
\usepackage{hyperref}
\usepackage{color}
\usepackage{soul}

%%%%%%%%%%%---SETME-----%%%%%%%%%%%%%
\newcommand{\hpcasubmissionnumber}{XXX}
%%%%%%%%%%%%%%%%%%%%%%%%%%%%%%%%%%%%
%\sethlcolor{white}

% When sethlcolor is white, your highlights will not show up.  Use
% \sethlcolor{white} to submit your paper pdf.  When compiling your second
% pdf with highlighted changes, simply remove \sethlcolor{white} and add your
% optional 100-word appendix.
% Use \hl{ ... } to highlight any text.
%%%%%%%%%%%%%%%%%%%%%%%%%%%%%%%%%%%%

\fancypagestyle{firstpage}{
  \fancyhf{}
\setlength{\headheight}{50pt}
\renewcommand{\headrulewidth}{0pt}
  \fancyhead[C]{\normalsize{HPCA 2020 Submission
      \textbf{\#\hpcasubmissionnumber} -- Confidential Draft -- Do NOT Distribute!!}}
  \pagenumbering{arabic}
}

%%%%%%%%%%%---SETME-----%%%%%%%%%%%%%
\title{CRUCIBLE: Computationally-RedUctible Complex Inline AccelerataBLE Units}
%%%%%%%%%%%%%%%%%%%%%%%%%%%%%%%%%%%%

\begin{document}
\maketitle
\thispagestyle{firstpage}
\pagestyle{plain}



%%%%%% -- PAPER CONTENT STARTS-- %%%%%%%%

\begin{abstract}

Computer and system architecture is currently experiencing a revolution. The dramatic
increase in workload variety and the usage of accelerators is shifting the paradigm from general-purpose
processors (GPP) providing the bulk of the performance, to more specialized units (i.e. accelerators) with tailored
design and specific to the task at which they excel. This has lead to the rise of new System-on-Chip (SoC) architectures
where GPP are complimented with a variety of heterogeneous accelerators. However, developing new hardware specific
for each application is a costly endeveour not only in development costs but also verification costs. More over, which
code region should be accelerated is of paramount importance to the designer, as each accelerator is tailored to a particular code region (i.e. functionality).
In this paper we present CRUCIBLE, an end-to-end methodology to derive Computationally-RedUctible Complex Inline AccelerataBLE Units
tailored to each application providing to the system designer an agile, systematic and cheap procedure to derive area-efficient by merging 
accelerator functionality, which does not only provide performance benefits with low area overhead, but is easily verifiable and highly integratable as they 
are part of the core.

\end{abstract}

\section{Introdution}



\section{Acknowledgements}


%%%%%%% -- PAPER CONTENT ENDS -- %%%%%%%%


%%%%%%%%% -- BIB STYLE AND FILE -- %%%%%%%%
\bibliographystyle{ieeetr}
\bibliography{ref}
%%%%%%%%%%%%%%%%%%%%%%%%%%%%%%%%%%%%

\end{document}
